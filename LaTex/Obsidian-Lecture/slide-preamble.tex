\usepackage{kotex}
\usepackage{commath}
\usepackage{multirow}
\usepackage{multicol}
\usepackage{arydshln} % Include this package
\usepackage{bbding}

\usepackage{tikz}
\usepackage{tikz-cd}
\usetikzlibrary{math} % for calculations
\usetikzlibrary{shapes, arrows.meta, positioning, shadows}
\usetikzlibrary{arrows,shapes.geometric}  % Optional, based on your needs
\usetikzlibrary{3d, calc}
\usetikzlibrary{matrix, positioning, arrows.meta, shapes.multipart, chains}
\usetikzlibrary{decorations.pathreplacing,calligraphy}
\usetheme[progressbar=frametitle]{metropolis}
\usepackage{appendixnumberbeamer}

\usepackage{adjustbox}
\usepackage{booktabs}
\usepackage[scale=2]{ccicons}

\usepackage{tcolorbox}

\usepackage{pgfplots}
\usepgfplotslibrary{fillbetween}
\pgfplotsset{compat=1.15}
\usepgfplotslibrary{dateplot}

\usepackage{xspace}
\newcommand{\themename}{\textbf{\textsc{metropolis}}\xspace}

\usepackage[linesnumbered,ruled]{algorithm2e}
\usepackage{algpseudocode}
\usepackage{setspace}
\SetKwComment{Comment}{/* }{ */}
\SetKwProg{Fn}{Function}{:}{end}
\SetKw{End}{end}
\SetKw{DownTo}{downto}

% Define a new environment for algorithms without line numbers
\newenvironment{algorithm2}[1][]{
	% Save the current state of the algorithm counter
	\newcounter{tempCounter}
	\setcounter{tempCounter}{\value{algocf}}
	% Redefine the algorithm numbering (remove prefix)
	\renewcommand{\thealgocf}{}
	\begin{algorithm}
	}{
	\end{algorithm}
	% Restore the algorithm counter state
	\setcounter{algocf}{\value{tempCounter}}
}

\usepackage{xcolor}   % Required for specifying colors

\usepackage{listings} %Code
\renewcommand{\lstlistingname}{Code}%
\definecolor{obsidian}{HTML}{8A77DE}
\definecolor{keyword}{RGB}{255, 0, 0}
\definecolor{identifier}{RGB}{0, 0, 255}
\definecolor{comment}{RGB}{0, 128, 0}
\definecolor{string}{RGB}{163, 21, 21}

\lstdefinestyle{c}{
	language=C,
	basicstyle=\ttfamily\small,
	keywordstyle=\color{keyword},
	identifierstyle=\color{identifier},
	commentstyle=\color{comment}\itshape,
	stringstyle=\color{string},
	showstringspaces=false,
	%	numberstyle=\tiny\color{gray},
	%	numbersep=5pt,
	frame=single,
	tabsize=4,
	captionpos=b,
	breaklines=true,
	breakatwhitespace=true,
	%	numbers=left
}

%% Define Markdown language (optional, if you want specific rules)
%\lstdefinelanguage{Markdown}{
%	morekeywords={#, *, -, `, [[, ]], ![}, % Custom Markdown symbols
%	sensitive=true,
%	morecomment=[l]{//}, % Define comments in Markdown
%	morestring=[b]"
%}

% Define a custom style for Markdown
%\lstdefinestyle{myMK}{
%	language=Markdown,
%	basicstyle=\ttfamily\small, % Font style for the code block
%	keywordstyle=\color{blue}, % Color for keywords
%	stringstyle=\color{orange}, % Color for strings
%	commentstyle=\color{green}, % Color for comments
%	backgroundcolor=\color{lightgray!20}, % Background color for the code block
%	frame=single, % Frame around the code block
%	breaklines=true, % Line breaking in long lines
%	showstringspaces=false, % Don't show spaces as special characters
%}

\definecolor{commentColor}{rgb}{0.25,0.5,0.35}
\definecolor{stringColor}{rgb}{0.5,0.1,0.5}
\definecolor{vsCodeRed}{HTML}{E06C6A}
\definecolor{vsCodeBlue}{HTML}{3986E9}
\definecolor{CLIGreen}{HTML}{87D368}
\definecolor{rustBuild}{HTML}{A5BE5D}
\definecolor{rustBack}{HTML}{282C34}
\lstdefinestyle{zsh}{
	language=bash,                  % Set the language to bash (closest to Zsh)
	backgroundcolor=\color{rustBack},
	commentstyle=\color{rustBuild}\ttfamily,
	keywordstyle=\color{rustBuild}\bfseries,
	stringstyle=\color{stringColor!70!white}\ttfamily,
	keywordstyle=[2]{\bfseries\color{rustBuild}},
	keywordstyle=[3]{\bfseries\color{CLIGreen}},
	showspaces=false,               % Don't show spaces as underscores
	showstringspaces=false,         % Don't highlight spaces in strings
	breaklines=true,                % Automatic line breaking
	frame=none,                     % No frame around the code
	basicstyle=\ttfamily\color{white}, % White basic text color for contrast
	extendedchars=true,             % Allow extended characters
	captionpos=b,                   % Caption-position at bottom
	escapeinside={\%*}{*)},         % Allow LaTeX inside your code
	morekeywords={echo,ls,cd,pwd,exit,clear,man,unalias,zsh,source}, % Add more keywords
	upquote=true,                   % Ensure straight quotes are used
	literate=
	{\$}{{\textcolor{vsCodeRed}{$\boldsymbol{\$}$}}}1
	{>}{{\textcolor{CLIGreen}{$\boldsymbol{>}$}}}1
	{<}{{\textcolor{CLIGreen}{$\boldsymbol{<}$}}}1
	{@}{{\textcolor{vsCodeRed}{@}}}1
	{~}{{\textcolor{vsCodeRed}{$\boldsymbol{\sim}$}}}1
	{:}{{\textcolor{rustBuild}{:}}}1,
	%		{:}{{\textcolor{red}{:}}}1
	%		{~}{{\textcolor{red}{\textasciitilde}}}1 % Color certain characters
	%		{├}{{\textbrokenbar}}1  % Replace ├ with \textbrokenbar
	%		{─}{{\textendash}}1     % Replace ─ with \textendash
	%		{└}{{`}}1,              % Replace └ with `
	morekeywords=[2]{cryptol, load, prove, ocaml},
	morekeywords=[3]{Main}
}

\usepackage{amsthm, amsmath}


\newcommand{\cyclic}[1]{\langle #1 \rangle}
\newcommand{\uniform}{\overset{\$}{\leftarrow}}
\newcommand{\N}{\mathbb{N}}
\newcommand{\Z}{\mathbb{Z}}
\newcommand{\Q}{\mathbb{Q}}
\newcommand{\R}{\mathbb{R}}
\newcommand{\C}{\mathbb{C}}
\newcommand{\F}{\mathbb{F}}

\newcommand{\xmark}{\textcolor{red}{\XSolidBrush}}
\newcommand{\vmark}{\textcolor{green!75!black}{\CheckmarkBold}}
