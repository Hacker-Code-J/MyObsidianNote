\documentclass[10pt, xcolor=dvipsnames]{beamer}
\usepackage{kotex}
\usepackage{commath}
\usepackage{multirow}
\usepackage{multicol}
\usepackage{arydshln} % Include this package
\usepackage{bbding}

\usepackage{tikz}
\usepackage{tikz-cd}
\usetikzlibrary{math} % for calculations
\usetikzlibrary{shapes, arrows.meta, positioning, shadows}
\usetikzlibrary{arrows,shapes.geometric}  % Optional, based on your needs
\usetikzlibrary{3d, calc}
\usetikzlibrary{matrix, positioning, arrows.meta, shapes.multipart, chains}
\usetikzlibrary{decorations.pathreplacing,calligraphy}
\usetheme[progressbar=frametitle]{metropolis}
\usepackage{appendixnumberbeamer}

\usepackage{adjustbox}
\usepackage{booktabs}
\usepackage[scale=2]{ccicons}

\usepackage{tcolorbox}

\usepackage{pgfplots}
\usepgfplotslibrary{fillbetween}
\pgfplotsset{compat=1.15}
\usepgfplotslibrary{dateplot}

\usepackage{xspace}
\newcommand{\themename}{\textbf{\textsc{metropolis}}\xspace}

\usepackage[linesnumbered,ruled]{algorithm2e}
\usepackage{algpseudocode}
\usepackage{setspace}
\SetKwComment{Comment}{/* }{ */}
\SetKwProg{Fn}{Function}{:}{end}
\SetKw{End}{end}
\SetKw{DownTo}{downto}

% Define a new environment for algorithms without line numbers
\newenvironment{algorithm2}[1][]{
	% Save the current state of the algorithm counter
	\newcounter{tempCounter}
	\setcounter{tempCounter}{\value{algocf}}
	% Redefine the algorithm numbering (remove prefix)
	\renewcommand{\thealgocf}{}
	\begin{algorithm}
	}{
	\end{algorithm}
	% Restore the algorithm counter state
	\setcounter{algocf}{\value{tempCounter}}
}

\usepackage{xcolor}   % Required for specifying colors

\usepackage{listings} %Code
\renewcommand{\lstlistingname}{Code}%
\definecolor{keyword}{RGB}{255, 0, 0}
\definecolor{identifier}{RGB}{0, 0, 255}
\definecolor{comment}{RGB}{0, 128, 0}
\definecolor{string}{RGB}{163, 21, 21}

\lstdefinestyle{c}{
	language=C,
	basicstyle=\ttfamily\small,
	keywordstyle=\color{keyword},
	identifierstyle=\color{identifier},
	commentstyle=\color{comment}\itshape,
	stringstyle=\color{string},
	showstringspaces=false,
	%	numberstyle=\tiny\color{gray},
	%	numbersep=5pt,
	frame=single,
	tabsize=4,
	captionpos=b,
	breaklines=true,
	breakatwhitespace=true,
	%	numbers=left
}

\definecolor{backcolor}{HTML}{FAFAFA}
\lstdefinestyle{ocaml}{
	language=[Objective]Caml,
	basicstyle=\ttfamily\bfseries\footnotesize,
	keywordstyle=\color{blue}\bfseries,
	commentstyle=\color{gray}\itshape,
	stringstyle=\color{orange}\bfseries,
	numberstyle=\tiny\color{purple},
	identifierstyle=\color{teal},
	emphstyle=\color{red}\bfseries,
	backgroundcolor=\color{backcolor},
	frame=single,
	rulecolor=\color{black},
	frameround=tttt,
	numbers=left,
	numbersep=10pt,
	showspaces=false,
	showstringspaces=false,
	breaklines=true,
	breakatwhitespace=true,
	tabsize=2,
	captionpos=b,
	literate={->}{{$\rightarrow$}}1
	{<-}{{$\leftarrow$}}1
	{=>}{{$\Rightarrow$}}1
	{|->}{{$\mapsto$}}1
	{>>}{{$\gg$}}1
	{<<}{{$\ll$}}1
}

\definecolor{commentColor}{rgb}{0.25,0.5,0.35}
\definecolor{stringColor}{rgb}{0.5,0.1,0.5}
\definecolor{vsCodeRed}{HTML}{E06C6A}
\definecolor{vsCodeBlue}{HTML}{3986E9}
\definecolor{CLIGreen}{HTML}{87D368}
\definecolor{rustBuild}{HTML}{A5BE5D}
\definecolor{rustBack}{HTML}{282C34}
\lstdefinestyle{zsh}{
	language=bash,                  % Set the language to bash (closest to Zsh)
	backgroundcolor=\color{rustBack},
	commentstyle=\color{rustBuild}\ttfamily,
	keywordstyle=\color{rustBuild}\bfseries,
	stringstyle=\color{stringColor!70!white}\ttfamily,
	keywordstyle=[2]{\bfseries\color{rustBuild}},
	keywordstyle=[3]{\bfseries\color{CLIGreen}},
	showspaces=false,               % Don't show spaces as underscores
	showstringspaces=false,         % Don't highlight spaces in strings
	breaklines=true,                % Automatic line breaking
	frame=none,                     % No frame around the code
	basicstyle=\ttfamily\color{white}, % White basic text color for contrast
	extendedchars=true,             % Allow extended characters
	captionpos=b,                   % Caption-position at bottom
	escapeinside={\%*}{*)},         % Allow LaTeX inside your code
	morekeywords={echo,ls,cd,pwd,exit,clear,man,unalias,zsh,source}, % Add more keywords
	upquote=true,                   % Ensure straight quotes are used
	literate=
	{\$}{{\textcolor{vsCodeRed}{$\boldsymbol{\$}$}}}1
	{>}{{\textcolor{CLIGreen}{$\boldsymbol{>}$}}}1
	{<}{{\textcolor{CLIGreen}{$\boldsymbol{<}$}}}1
	{@}{{\textcolor{vsCodeRed}{@}}}1
	{~}{{\textcolor{vsCodeRed}{$\boldsymbol{\sim}$}}}1
	{:}{{\textcolor{rustBuild}{:}}}1,
	%		{:}{{\textcolor{red}{:}}}1
	%		{~}{{\textcolor{red}{\textasciitilde}}}1 % Color certain characters
	%		{├}{{\textbrokenbar}}1  % Replace ├ with \textbrokenbar
	%		{─}{{\textendash}}1     % Replace ─ with \textendash
	%		{└}{{`}}1,              % Replace └ with `
	morekeywords=[2]{cryptol, load, prove, ocaml},
	morekeywords=[3]{Main}
}

\usepackage{amsthm, amsmath}


\newcommand{\cyclic}[1]{\langle #1 \rangle}
\newcommand{\uniform}{\overset{\$}{\leftarrow}}
\newcommand{\N}{\mathbb{N}}
\newcommand{\Z}{\mathbb{Z}}
\newcommand{\Q}{\mathbb{Q}}
\newcommand{\R}{\mathbb{R}}
\newcommand{\C}{\mathbb{C}}
\newcommand{\F}{\mathbb{F}}

\newcommand{\xmark}{\textcolor{red}{\XSolidBrush}}
\newcommand{\vmark}{\textcolor{green!75!black}{\CheckmarkBold}}

\newcommand{\B}{\mathbb{B}}
\newcommand{\true}{\textcolor{red}{\texttt 1}}
\newcommand{\false}{\textcolor{red}{\texttt 0}}
\newcommand{\id}{\textnormal{id}}
\title{\huge\bf \textcolor{obsidian}{Obsidian} for Researchers}
%\subtitle{\textcolor{magenta}{\textbf{Lecture 02. OCaml Programming II}}}
% \date{\today}
\date{}
\author{\large\textcolor{cyan}{\bf Ji, Yong-Hyeon}\\ \\ \small 24. 09. 10 (Thu)}
\institute{\small
	Coding \& Optimization Together (CO2) \\
	Crypto \& Security Engineering Lab (CSE) \\
	Department of Information Security, Cryptology, and Mathematics
}
%\titlegraphic{\hfill\includegraphics[height=1.5cm]{../latex-imagelogo}}

\pgfdeclareimage[height=\paperheight,width=\paperwidth]{myimage}{../latex-image/hexagon_bg}
\usebackgroundtemplate{\tikz\node[opacity=0.5] {\pgfuseimage{myimage}};}

%\setbeamercolor{title}{bg=UniBlue}
\setbeamercolor{frametitle}{bg=obsidian}
%\setbeamercolor{structure}{bg=UniBlue}

\begin{document}
	\maketitle\begin{frame}{}
	\begin{figure}[\centering]
		\includegraphics[scale=.3]{../latex-image/obsidian-icon}
	\end{figure}
	\end{frame}
	\begin{frame}{Table of Contents}
		\setbeamertemplate{section in toc}[sections numbered]
		\tableofcontents%[hideallsubsections]
	\end{frame}
	
	\newpage
	\section{section1}
	\begin{frame}{2.1 OCaml 기본 구성}
		\textbf{$\triangleright$ \textcolor{red}{Function Expression} (함수식)}
		\[
		\texttt{fun $x$ -> $e$}
		\] 
		\begin{itemize}
			\item 함수의 예:
			\begin{itemize}
				\item[*] \texttt{fun $x$ -> $x+1$}
				\item[*] \texttt{fun $y$ -> $y*y$}
				\item[*] \texttt{fun $x$ -> if $x>0$ then $x+1$ else $x*x$}
				\item[*] \texttt{fun $x$ -> fun $y$ -> $x+y$}
				\item[*] \texttt{fun $x$ -> fun $y$ -> fun $z$ -> $x+y+z$}
			\end{itemize}
			\item[]
			\item Syntactic Sugar \[
			\texttt{fun $x_1\ \dots\ x_n$\ ->\ $e$}
			\]
			\begin{itemize}
				\item[*] \texttt{fun $x$ $y$ -> $x+y$}
				\item[*] \texttt{fun $x$ $y$ $z$ -> $x+y+z$}
			\end{itemize}
		\end{itemize}
	\end{frame}
	\begin{frame}{2.1 OCaml 기본}
		\textbf{$\triangleright$ Function Call Expression (함수 호출식)} \[
		e_1\quad e_2
		\]
		\begin{tcolorbox}[colback=backcolor]\ttfamily
			\# (fun x -> x * x) 3;;\\
			- : int = 9\\
			\# (fun x -> if x > 0 then x + 1 else x * x) 1;;\\
			- : int = 2\\
			\# (fun x -> fun y -> fun z -> x + y + z) 1 2 3;;\\
			- : int = 6
		\end{tcolorbox}
		
		\begin{tcolorbox}[colback=backcolor]\ttfamily
			\# (fun f -> f * 1) (fun x -> x * x);;\\
			- : int = 1\\
			\# (fun x -> x * x) ((fun x -> if x > 0 then 1 else 2) 3);;\\
			- : int = 2
		\end{tcolorbox}
	\end{frame}

	\begin{frame}{2.1 OCaml 기본}
		\textbf{$\triangleright$ Let Expressions}
		
		값에 이름 붙이기! \[
		\texttt{let}\ x = e_1\ \texttt{in}\ e_2
		\]
		\begin{itemize}
			\item $e_1$의 값을 $x$라고 하고 $e_2$를 계산
			\begin{itemize}
				\item[*] $x$: variable (변수, 값의 이름)
				\item[*] $e_1$: binding expression (정의식)
				\item[*] $e_2$: body expression (몸통식)
			\end{itemize}
			\item $e_2$: scope of $x$ (유효범위)
		\end{itemize}
		 \begin{tcolorbox}[colback=backcolor]\ttfamily
			\# let x = 1 in x + x;;\\
			- : int = 2\\
			\# (let x = 1 in x) + x;;\\
			Error: Unbound value x\\
			\# (let x = 1 in x) + (let x = 2 in x);; \\
			- : int = 3
		\end{tcolorbox}	
		\begin{itemize}
			\item[] 
			\item[] 
			\item[] 
			\item[] 
			\item[] 
		\end{itemize}
	\end{frame}

\begin{lstlisting}[style=zsh]
# let x = (let y = 1 in y + 1) in x + 1;;
- : int = 3
# let x = 1 in
	let y = 2 in
		x + y;;
- : int = 3
\end{lstlisting}

\begin{lstlisting}[style=zsh]
# let square = fun x -> x * x in square 2;;
- : int = 4
# let add x y = x + y in add 1 2;;
- : int = 3
\end{lstlisting}

\begin{lstlisting}[style=zsh]
# let rec factorial n =
	if n = 0 then 1
	else n * factorial (n - 1);;    
val factorial : int -> int = <fun>
# factorial 5;;
- : int = 120
\end{lstlisting}
	\newpage
	\begin{frame}{2.1 OCaml 기본}
		\textbf{$\triangleright$ Pattern Matching (패턴 매칭)}
		
		\begin{itemize}
			\item 패턴 매칭을 이용한 값의 구조 분석
		\end{itemize}
		\begin{tcolorbox}[colback=backcolor]\ttfamily
		\# let rec factorial n =\\
			if n = 0 then 1 else n * factorial (n - 1);; \\   
		val factorial : int -> int = <fun>
		\end{tcolorbox}
		\begin{tcolorbox}[colback=backcolor]\ttfamily
			\# let factorial a = \\
			match a with\\
			0 -> 1\\
			|\_ -> a * factorial (a-1);;\\
			val factorial : int -> int = <fun>
		\end{tcolorbox}
		\begin{itemize}
			\item[]
			\item[]
		\end{itemize}
	\end{frame}

	\begin{frame}{2.1 OCaml 기본}
		\textbf{$\triangleright$ Polymorphic Type (다형 타입)}
		
		\begin{tcolorbox}[colback=backcolor]\ttfamily
			\# let id x = x;; \\
			val id : 'a -> 'a = <fun> \\
			\# id 1;; \\
			- : int = 1 \\
			\# id "abc";; \\
			- : string = "abc" \\
			\# id true;; \\
			- : bool = true
		\end{tcolorbox}
		\begin{itemize}
			\item[]
			\item[]
			\item[]
			\item[]
		\end{itemize}
	\end{frame}
	
	\newpage
	{\setbeamercolor{palette primary}{fg=black, bg=-blue}
		\begin{frame}[standout]
			To be continue ...
		\end{frame}
	}
	
	%\appendix
	%
	%\begin{frame}[fragile]{Backup slides}
	%  Sometimes, it is useful to add slides at the end of your presentation to
	%  refer to during audience questions.
	%
	%  The best way to do this is to include the \verb|appendixnumberbeamer|
	%  package in your preamble and call \verb|\appendix| before your backup slides.
	%
	%  \themename will automatically turn off slide numbering and progress bars for
	%  slides in the appendix.
	%\end{frame}
	%
	%\begin{frame}[allowframebreaks]{References}
	%
	%  \bibliography{demo}
	%  \bibliographystyl{abbrv}
	%
	%\end{frame}
	
\end{document}
